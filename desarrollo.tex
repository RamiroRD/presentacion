%------------------------------------------------
\section{Introducción} 
%------------------------------------------------

\subsection{Descripción}
 
%%%% IDEAS %%%%
% Dos renglones
% Rojo, Naranja o verde
% Bloque de 7x5 minimo.. entre 10 y 15 bloques (letras) 
% Modular la velocidad
% Juego mario, dinosaurio chrome

\begin{frame}
	\frametitle{Descripción}
	
		\begin{block}{Motivación}
			\begin{itemize}
				\item Alto costo de productos en el mercado
				\item Reducción del uso del papel
			\end{itemize}
		\end{block}

		\begin{block}{Características generales}
			\begin{itemize}
				\item Caracteres de 7x5 píxeles
				\item Dos renglones de 10-15 caracteres
				\item Control por WiFi
				\item Animación programable
			\end{itemize}
		\end{block}

\end{frame}

\subsection{Objetivos}
\begin{frame}
	\frametitle{Objetivos}
	\begin{block}{Objetivos primarios}
		\begin{itemize}
			\item Diseñar el Hardware (PCB, Cartel)
			\item Escribir el Software que se ejecutará en el microcontrolador
			\item Armar prototipo funcional
		\end{itemize}
	\end{block}
	
	\begin{block}{Objetivo secundario}
		\begin{itemize}
			\item Escribir biblioteca para la CIAA que abstraiga comandos AT para módulo Espressif ESP8266
		\end{itemize}
	\end{block}
\end{frame}

%------------------------------------------------
\section{Especificación}
%------------------------------------------------

\subsection{Requerimientos}
\begin{frame}
	\frametitle{Requerimientos}
	\begin{block}{Hardware}
		\begin{itemize}
			\item 30 chips MAX1719 o MAX1721
			\item 30 matices 8x8 1388ASR o 1588AS
			\item Un microcontrolador EDU-CIAA
			\item Modulo ESP8266
		\end{itemize}
	\end{block}
	\begin{block}{Software}
		\begin{itemize}
			\item Host con capacidad para bootstrap 4
			% Y algun lenguaje para hacer el sistema de autentificacion
			\item 
		\end{itemize}
	\end{block}
	

\end{frame}

\subsection{Diagrama en bloques del sistema}
\begin{frame}
	\frametitle{Diagrama en bloques}

\end{frame}

\subsection{Explicación funcional del HW y SW}
\begin{frame}
	\frametitle{Explicación funcional}
	\begin{itemize}
		 \item El sistema se conecta una red WiFi preexistente y hostea un panel de configuración accesible a través de la red
		 \item El usuario accede al  panel mediante HTTPS y se autentica
		 \item El usuario ingresa el anuncio a mostrar y las opciones deseadas
		 \item El micro recibe el comando y enciende los LEDs de manera que muestren el mensaje
		 \item El mensaje se desplaza a lo largo del cartel a modo de marquesina
	\end{itemize}
	
	% El microcontrolador que maneja el cartel compuesto de LED's, es el encargado de establecer una red WIFI (sin acceso a internet) sobre la cual hostea un panel de configuración. Por medio de un usuario y clave, se accede al panel vía https y se puede cambiar el anuncio a representar en el cartel. El mensaje a representar, se desplaza a lo largo de la matriz de LED's a modo de marquesina y, dentro del panel, se pueden configurar determinados parámetros como por ejemplo, la velocidad a la que se mueve.

\end{frame}

%------------------------------------------------
\section{Gestión}
%------------------------------------------------

\subsection{Presupuesto}
\begin{frame}
	\frametitle{Presupuesto}
	\FPeval{\cantLEDs}{round( 2*15 , 0)}
	\begin{table}[]
		\centering
		\begin{spreadtab}{{tabular}{cccc}}
			@ Item				& @ Unidades	& @ Precio unidad	& @ Subtotal	\\ \hline
			@ MAX1719 \& Matriz	& \cantLEDs		& 70				& b2*c2	\\
			@ Cables			& :={2} m		& 2					& b3*c3  \\
			@ Estaño          	& 1				& 100				& b4*c4  \\
			@ EDU-CIAA			& 1				& 1000				& b5*c5  \\ \hline
			@ Total				& 				&					& sum(d2:d5)	 \\ \hline
		\end{spreadtab}
	\end{table}
\end{frame}

\subsection{Cronograma}
\begin{frame}
	\frametitle{Cronograma}
	\begin{table}[]
		\centering
		\caption{Cronograma primera etapa}
		\begin{tabular}{cccc}
			Encargado & Descripción   & Fecha inicial & Fecha final \\ \hline
			Ternouski & Tarea Hw 1    & 12/09/17      & 12/12/17    \\
					  & Tarea  2      & Hw            & 1           \\
					  &               &               &             \\
					  &               &               &             \\
					  &               &               &             \\
					  &               &               &             \\ \hline
		\end{tabular}
	\end{table}
\end{frame}

\subsection{División de tareas del grupo}
\begin{frame}
	\frametitle{División de tareas del grupo}

\end{frame}

%------------------------------------------------
\section{Referencias}
%------------------------------------------------

\begin{frame}
	\frametitle{Referencias}
	\footnotesize{
	\begin{thebibliography}{99}
		\bibitem[Nielsen, 2001]{ref:Nielsen} Nielsen, Jakob. (2001)
		\newblock Beyond Accessibility: Treating People with Disabilities as People.
		\newblock \emph{Alertbox}.
		Disponible en:
		\href{http://www.useit.com/alertbox/20011111.html}{www.useit.com}
	\end{thebibliography}
	}
\end{frame}