%------------------------------------------------
\section{Introducción} 
%------------------------------------------------

\subsection{Descripción}
 
%%%% IDEAS %%%%
% Dos renglones
% Rojo, Naranja o verde
% Bloque de 7x5 minimo.. entre 10 y 15 bloques (letras) 
% Mudular la velocidad
% Juego mario, dinosaurio chrome

\begin{frame}
	\frametitle{Descripción}
	
		\begin{block}{Motivación}
			Precio alto, uso dentro de la facultad para minimizar la cantidad de papeles.
		\end{block}

		\begin{block}{Características generales}
			Lorem ipsum dolor sit amet, consectetur adipiscing elit. Integer lectus nisl, ultricies in feugiat rutrum, porttitor sit amet augue. Aliquam ut tortor mauris. Sed volutpat ante purus, quis accumsan dolor.
		\end{block}

\end{frame}

\subsection{Objetivos}
\begin{frame}
	\frametitle{Objetivos}

\end{frame}

%------------------------------------------------
\section{Especificación}
%------------------------------------------------

\subsection{Requerimientos}
\begin{frame}
	\frametitle{Requerimientos}

\end{frame}

\subsection{Diagrama en bloques del sistema}
\begin{frame}
	\frametitle{Diagrama en bloques}

\end{frame}

\subsection{Explicación funcional del HW y SW}
\begin{frame}
	\frametitle{Explicación funcional}

\end{frame}

%------------------------------------------------
\section{Gestión}
%------------------------------------------------

\subsection{Presupuesto}
\begin{frame}
	\frametitle{Presupuesto}

\end{frame}

\subsection{Cronograma}
\begin{frame}
	\frametitle{Cronograma}
		\begin{table}[]
			\centering
			\caption{Cronograma primera etapa}
			\label{my-label}
			\begin{tabular}{cccc}
				Encargado & Descripción & Fecha inicial & Fecha final \\ \hline
				Ternouski & Tarea Hw 1  & 12/09/17       & 12/12/17    \\
 				          & Tarea  2 dsd     & Hw            & 1           \\
				          &             &               &             \\
				          &             &               &             \\
				          &             &               &             \\
				          &             &               &             \\ \hline
\end{tabular}
\end{table}
\end{frame}

\subsection{División de tareas del grupo}
\begin{frame}
	\frametitle{División de tareas del grupo}

\end{frame}

%------------------------------------------------
\section{Referencias}
%------------------------------------------------

\begin{frame}
	\frametitle{Referencias}
	\footnotesize{
	\begin{thebibliography}{99}
		\bibitem[Nielsen, 2001]{ref:Nielsen} Nielsen, Jakob. (2001)
		\newblock Beyond Accessibility: Treating People with Disabilities as People.
		\newblock \emph{Alertbox}.
		Disponible en:
		\href{http://www.useit.com/alertbox/20011111.html}{www.useit.com}
	\end{thebibliography}
	}
\end{frame}