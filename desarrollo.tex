%------------------------------------------------
\section{Introducción} 
%------------------------------------------------

\subsection{Descripción}
 
%%%% IDEAS %%%%
% Dos renglones
% Rojo, Naranja o verde
% Bloque de 7x5 minimo.. entre 10 y 15 bloques (letras) 
% Mudular la velocidad
% Juego mario, dinosaurio chrome

\begin{frame}
	\frametitle{Descripción}
	
		\begin{block}{Motivación}
			Precio alto de adquisicion , uso dentro de la facultad para minimizar la cantidad de papeles.
		\end{block}

		\begin{block}{Características generales}
			El cartel tendra dos rengones para poder mostrar dos mensajes independientes, con una resolución de 7x5 LEDs por caractér como minimo y con una longitud aproximada de entre 10 y 15 caracteres por renglón.
		\end{block}

\end{frame}

\subsection{Objetivos}
\begin{frame}
	\frametitle{Objetivos}

\end{frame}

%------------------------------------------------
\section{Especificación}
%------------------------------------------------

\subsection{Requerimientos}
\begin{frame}
	\frametitle{Requerimientos}

\end{frame}

\subsection{Diagrama en bloques del sistema}
\begin{frame}
	\frametitle{Diagrama en bloques}

\end{frame}

\subsection{Explicación funcional del HW y SW}
\begin{frame}
	\frametitle{Explicación funcional}

\end{frame}

%------------------------------------------------
\section{Gestión}
%------------------------------------------------

\subsection{Presupuesto}
\begin{frame}
	\frametitle{Presupuesto}
	\FPeval{\cantLEDs}{round( 2*15*7*5 , 0)}
	\begin{table}[]
		\centering
		\begin{spreadtab}{{tabular}{cccc}}
			@ Item				& @ Unidades	& @ Precio unidad	& @ Subtotal	\\ \hline
			@ LED color X		& \cantLEDs		& 0.5				& b2*c2	\\
			@ Cables			& :={2} m		& 2					& b3*c3  \\
			@ Estaño p/ soldar	& 1				& 100				& b4*c4  \\
			@ Chip X			& 2*15			& 10				& b5*c5  \\
			@ Edu-ciaa			& 1				& 1000				& b6*c6  \\ \hline
			@ Total				& 				&					& sum(d2:d6)	 \\ \hline
		\end{spreadtab}
	\end{table}
\end{frame}

\subsection{Cronograma}
\begin{frame}
	\frametitle{Cronograma}
	\begin{table}[]
		\centering
		\caption{Cronograma primera etapa}
		\begin{tabular}{cccc}
			Encargado & Descripción & Fecha inicial & Fecha final \\ \hline
			Ternouski & Tarea Hw 1  & 12/09/17       & 12/12/17    \\
						& Tarea  2 dsd     & Hw            & 1           \\
						&             &               &             \\
						&             &               &             \\
						&             &               &             \\
						&             &               &             \\ \hline
		\end{tabular}
	\end{table}
\end{frame}

\subsection{División de tareas del grupo}
\begin{frame}
	\frametitle{División de tareas del grupo}

\end{frame}

%------------------------------------------------
\section{Referencias}
%------------------------------------------------

\begin{frame}
	\frametitle{Referencias}
	\footnotesize{
	\begin{thebibliography}{99}
		\bibitem[Nielsen, 2001]{ref:Nielsen} Nielsen, Jakob. (2001)
		\newblock Beyond Accessibility: Treating People with Disabilities as People.
		\newblock \emph{Alertbox}.
		Disponible en:
		\href{http://www.useit.com/alertbox/20011111.html}{www.useit.com}
	\end{thebibliography}
	}
\end{frame}